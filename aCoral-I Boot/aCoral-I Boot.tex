\documentclass{thesis-uestc}
% \usepackage{indentfirst} 
% \setlength{\parindent}{2em} %2em代表首行缩进两个字符

\title{珊瑚-I 启动过程}{aCoral-I Boot Process}
\author{王彬浩}{Wang BinHao}
\school{信息与软件工程学院}{School of Information and Software Engineering}
\major{电子信息}{Electrical & Information}
\studentnumber{202122090410}
\version{0.1}


% require all the usepackages here
% \usepackage{algorithm2e}
\usepackage{listings}
\usepackage{xcolor}
\usepackage{fontspec}
\usepackage{caption}
\usepackage{float}


\lstset{
    basicstyle=\small, 
    breaklines,                                 % 自动将长的代码行换行排版
    extendedchars=false,                        % 解决代码跨页时,章节标题,页眉等汉字不显示的问题
    backgroundcolor=\color[rgb]{0.96,0.96,0.96},% 背景颜色
    keywordstyle=\color{blue}\bfseries,         % 关键字颜色
    identifierstyle=\color{black},              % 普通标识符颜色
    commentstyle=\color[rgb]{0,0.6,0},          % 注释颜色
    stringstyle=\color[rgb]{0.58,0,0.82},       % 字符串颜色
    showstringspaces=false,                     % 不显示字符串内的空格
    numbers=left,                               % 显示行号
    captionpos=t,                               % title在上方(在bottom即为b)
    frame=single,                               % 设置代码框形式
    rulecolor=\color[rgb]{0.8,0.8,0.8},         % 设置代码框颜色
    tabsize=4
}  




\begin{document}

\makecover

% This is a template of mutiple files.
% The folders chapters/ and misc/ have the related files

\begin{revisionhistory}
    \begin{center}
        \setlength\tabcolsep{15pt}
        \begin{tabular}{|c|p{16em}<{\centering}|c|c|}
            \hline  版本号 & 内容 & 日期 & 负责人 \\
            \hline  0.1 & 开始编写,修改latex模板,确定大纲 & 2022.05.15 & 王彬浩 \\
            \hline 1.0 & 第一版aCoral启动手册编写完成 & 2022.06.12 & 王彬浩 \\
            \hline \Version & 更新acoral\_module\_init代码 & 2022.07.23 & 王彬浩 \\
            \hline
        \end{tabular}
    \end{center}



\end{revisionhistory}



% table of contents
\thesistableofcontents

% thesis contents
\chapter{Bootloader}
\section{什么是Bootloader}
我们将 Bootloader 拆开来看,
一个是 boot,一个是 loader,可见 Bootloader 有两个主要特性:

(1) Boot。
Boot的原意为靴子,在计算机领域引申为启动,也就是说系统启动时会从这里启动,具体一点就是当
我们按开机键,cpu 执行的第一条指令就应该是 Bootloader 的代码。

(2) Loader。
Loader是加载的意思。那自然就可以引出三个问题:加载什么(what)、怎么加载(how)、从哪里加载(from)、加载到哪里去(to)以及为什么要加载(why)。
加载什么?当然是加载代码程序,这个程序就是我们经常谈到的内核映像——像 linux 内核,window 内核,亦或是aCoral的内核等。
怎么加载?加载说白了就是复制,将内核代码原封不动的从一个地方复制到另一个地方。
从哪里加载?自然是从放着内核代码的地方加载了,常见的存储介质有 flash、SD 卡等。
加载到哪里去?自然是加载到内核要运行的地方——内存,常见的有 SDRAM、DDR 等。
为什么要加载?这就不是一句两句能说清楚的了。简单来讲,因为 flash、SD 卡这些非易失性存储器不能用来运行程序,或者不适合运行程序,但是他们便宜,容量大,断电也不会丢失数据,适合存放程序,我们一般称为外存。
而SDRAM、DDR这些易失性存储器不能用来在断电的情况下存放程序,但是在通电之后运行速度快,适合运行代码,一般称为内存。所以,上电之后需要通过 loader 来把内核程序从外存复制到内存中,这样内核程序就能正常运行了。

在嵌入式系统中,常见的Bootloader有 vivi,uboot ,这些程序都是开机时
就启动的,它启动后,会从 flash 或 sd 卡等存储设备中将内核程序代码拷贝
到 sdram,然后执行内核代码。

如果你曾经接触过vivi、uboot这些开源的bootloader,就会发现,这些bootloader似乎都没有这么简单,大小也都往往几MB往上,这是为什么呢?
有两个主要原因:
(1) 它们都支持多平台,它们都可以看成一个通用的 Bootloader。
(2) 除了提供上面启动,加载两个功能外,它们还支持更多功能,比如支持各种命令,比如操作 nandflash,norflash,EEPROM,支持 ftp,tftp,nfs 等网络协议,又或者支持 usb 下载等功能。
有些 Bootloader 如 arm 公司的 bootmonitor 还支持文件系统,能以文
件系统的方式管理 nandflash,sdcard,compact card 上的数据。
有了上面两大类的支持后,Bootloader 不再是纯粹的 Bootloader,都有了操
作系统的一些功能,只是不支持操作系统支持的任务切换功能。

aCoral的bootloader其实就是 
\begin{lstlisting}
 hal/s3c2440/src/start.S 
\end{lstlisting}
这个文件。它就没有这么多复杂的功能了,仅仅做了bootloader最本职的两个工作:启动、加载。关于这个文件的解析,将在后面详细阐述。


\section{loader}
关于为什么要加载(why)这个问题,这一小节就来细说。如果你对这一部分没有特别强烈的求知欲,可以暂时先跳过,以后再来阅读。
我们的程序代码存放或者运行的地方称为存储介质。储存介质选择的主要参考:速度,尺寸,价格。存储介质按照不同的方法,可以分为不同的种类。

1)按存取方式分类
如果存储器中任何存储单元的内容都能被随机存取,且存取时间和存储单元
的物理位置无关,这种存储器称为随机存储器。半导体存储器和磁芯存储器都是
随机存储器。如果存储器只能按某种顺序来存取,也就是说存取时间和存储单元
的物理位置无关,这种存储器称为顺序存储器。例如,磁带存储器就是顺序存储
器。一般来说,顺序存储器的存取周期较长。磁盘存储器是半顺序存储器。

2)按存储器的读写功能分类
有些半导体存储器存储的内容是固定不变的,即只能读出而不能写入,因此
这种半导体存储器称为只读存储器 (ROM)。既能读出又能写入的半导体存储器,
称为随机存储器 (RAM)。

3)按信息的可保存性分类
断电后信息即消失的存储器,称为易失性存储器,或者非永久记忆的存储器。断电后仍能保存信
息的存储器,称为非易失性存储器,或永久性记忆的存储器。磁性材料做成的存储器是永久性存储器,
半导体读写存储器 RAM 是非永久性存储器。

我们常见的储存介质大类有:磁带,硬盘,ROM,RAM,具体到嵌入式:经常用到是
norflash,nandflash,sdcard,TF 卡,compact 卡,sdram,ram 等。为什么会出现这
么多种类?这个是价格和需求平衡的结果。
比如,我们知道程序最后运行必须要有随机可读写存储器来存储变量,且速度
要快,这个导致了 RAM 的产生,但是 RAM 价格昂贵,又导致了 sdram 的产生,sdram
和 ram 的区别就是它是靠电容的值来保存 0,1 信息,时间一长就会丢失数据,故
需要周期性刷新,这个在 sdram 控制器芯片的控制下能很好解决,且不太影响性
能,但是它速度比 ram 低一些,且复杂些,但是价格低很多,且容易做到很大,
故是一种很好的储存器,因此目前无论是嵌入式还是 pc 设备都广泛使用到了sdram。
虽然 sdram 解决了可读写问题,且速度问题。但是它们都是非永久记忆的存
储器,断电后信息即消失的存储器,明显不能满足我们要求永久保存我们代码的
需求,你总不至于,每次启动电脑都要下载一次程序吧,于是就产生 nandflash,硬
盘这些永久记忆的存储器(硬盘太大,很少用在嵌入式系统中),这些储存器是永
久,且能做到很大容量,但是速度慢,不过还是可以承受的,因为我们有办法解
决这个问题?如何解决,就是前面说的加载,就是说在启动阶段,Bootloader 启动
后就从这些储存介质拷贝程序到 sdram,这样真正运行时,代码和数据是从 sdram
中读取的,也就没有速度问题了,这也是为啥要 Bootloader 的原因。

有了 nandflash,硬盘这些永久记忆的存储器还不够,为啥?因为它们是按块
访问的,而不是按地址访问,这种块模式访问往往需要有硬件控制器,而硬件控
制器又需要由程序来控制,那这个控制器的程序从何而来?这就是鸡生蛋,蛋生
鸡的问题,正因为这样又出来一种存储器——ROM,比如 norflash,只读存储器,
这种储存器也是永久记忆的存储器,但它和 nandflash 等不一样,它是按地址随
机访问的,也就是说不需要驱动,和 sdram 的访问方式一样,可以很简单的访问
数据,这就解决了这个问题,但是这种按地址访问的永久记忆的存储器相比有点
贵,且不能做到很大。其实也没必要过多的使用这中存储器,为啥?因为它是只
读的,没法修改,不会过多使用,所以只要能够容下 Bootloader 这些程序就可以
了,其他的代码交给廉价的可写的 nandflash 吧,当然对于代码还是可以一直放
在 rom 中的,这样可以减少 Sdam 的使用。

也许你会说为啥不出产一种按地址访问的可读写永久记忆的存储器,是可以
啊,但是代价太高,没必要,只要合理搭配,就可以满足需求,当然不排除有一
天,按地址访问的可读写永久记忆的存储器很便宜了,但是这个世界没有完美的
东西,优点越多,缺点也越多。

下面就来说下嵌入式储存器搭配问题:
硬盘肯定是不到万不得已,是不选择的,因为这个家伙体积大,功耗大,也
不安静,不过对于需要储存上 10G 的数据的应用,还不得不用它。
Bootloader 程序的储存器肯定得要是按地址随机存取的永久性记忆的存储器,
当然对于支持 nandflash 启动的 soc,也可以储存在 nandflash,比如 s3c2410,2440,
同时又比如 omap3530 是支持 sdcard 启动的,这样的 SOC 芯片也是可以将
Bootloader 放在 sdcard 上的。
也许到了这里,你会有一种强烈的好奇性?刚才不是说 nandflash,sdcard 都
是需要控制器才能访问数据的啊,控制器又需要程序,上面的 s3c2410,omap3530
等芯片是如何做到从这些地方启动的。
其实解决方法和我们上面探讨的一样,就是必须有一个拷贝动作,这个拷贝
动作可以有三种方式,一种是硬件方式,另一种是软件方式
1)硬件方式:
就是硬件实现储存设备控制器的控制,读取指定大小的数据,它没法做到控制
器的驱动程序那样,可以随机读取任意大小的数据,但是只要能够拷贝指定地址
指定大小的数据,就已经够了,硬件可以看成是简化版的驱动
2)软件方式: 
这个就更简单了,芯片自带一个 ROM,往往是片内 ROM,这个 ROM 里装有驱动程
序,这个驱动程序负责将我们的 Bootloader 从 nandflash 或 sdcard 等储存器拷
贝到 sdram 或 ram 后,然后跳到我们的 Bootloader 运行,这样其实和我们将
Bootloader 储存在 rom 是一样的,只不过板子自带了一个 Bootloader,这个简单
的 Bootloader 先于我们的 Bootloader 运行,主要实现小量数据(至少包括我们
的 Bootloader 的自我拷贝代码)拷贝。
其实还有另外一种启动技巧,那就是内存映射:
就是说当用户使用跳线选择方式后,硬件自动开启了内存映射,将其他内
存地址映射到 cpu 启动地址,比如 pb11mpcore,cpu 的启动地址是 0x0,如果配置
为 Norflash 启动,则可将 norflash 的地址 0x40000000~0x43FFFFFF 映射到
0x0~0x3ffffff,这样就相当于从 norflash 启动,这种方式是经常用的方式。由于
内存映射到地址 0x0 了,导致地址 0x0 对应的的内存没法使用,因此启动后需将
这个映射取消
这种和上面的方式不同,这种需要有按地址随机存取存储器的支持,即将储
存启动代码的存储器的地址映射到启动地址。

说完 Bootloader 的储存介质,就得说说操作系统(通常叫 kernel)映像文件
的储存介质。
嵌入式操作系统这类操作系统一般比较小,选择余地有很多,可以放在 rom
中,也可以放在 nandflash 中,因为不论放在哪里,只要 Bootloader 能找到,拷
贝到 sdram 就可以了,所以关键看 Bootloader 是否强大,对于很强大的 Bootloader,
其实操作系统都可以放在主机上,然后 Bootloader 可以通过网络将操作系统下载
到 sdram,然后启动操作系统。
对于 Bootloader 和内核链在一起的操作系统,操作系统肯定就是跟
Bootloader 一起储存在一种储存介质中了啊。

\chapter{aCoral启动}
2440上电之后,CPU将从0地址处开始取指执行指令。如果将aCoral程序存放在Norflash中,并通过mini2440开关S2选择Norflash启动,则Norflash从硬件层面上被映射为0地址开始的一段地址;
如果将aCoral放在Nandflash中,并且通过S2开关选择Nandflash启动,则开发板在上电后自动将Nandflash前4KB内容复制到开发板上的一块SRAM中。
这块SRAM我们称为Stepping Stone(垫脚石),并且Stepping Stone的地址就是从0地址开始。所以,不论选择何种启动方式,2440都将从0地址开始执行第一行代码。

图\ref{复位后S3C2440A的存储器映射}显示了复位后S3C2440A的存储器映射情况,详细请参考《S3C2440中文手册》第五章。
\begin{figure}[H]
	\includegraphics[width=0.9\textwidth]{复位后S3C2440A的存储器映射.png}
	\caption{复位后S3C2440A的存储器映射}
	\label{复位后S3C2440A的存储器映射}
\end{figure}

\section{启动-第一阶段}
之前说到,aCoral的bootloader其实就是
\begin{lstlisting}
 hal/s3c2440/src/start.S 
\end{lstlisting}

我们将其称为aCoral的启动文件。启动文件中,这行跳转指令就是整个aCoral的入口,将被烧录在Nandflash或Norflash的0地址。
\begin{lstlisting}
__ENTRY:
	b	ResetHandler
\end{lstlisting}

这句跳转程序将跳转到ResetHandler标号处,执行一些上电之后的硬件初始化工作,包括关闭看门狗、配置时钟、堆栈初始化、复制OS到SDRAM等。
我们一点点来看这些代码。

PS:请准备好《S3C2440中文手册》

\subsection{禁用看门狗}
\begin{lstlisting}
	@ disable watch dog timer
	mov r1, #0x53000000
	mov	r2, #0x0
	str	r2, [r1]
\end{lstlisting}

看门狗WatchDog的名字形象的描述了它的工作原理,看门狗每隔一段时间(比如:3个小时)它就会饥饿,每次饥饿时都叫,如果不想让它叫,只要我们保证在3个小时内喂狗一次就行。因此我们要及时的对看门狗控制器执行喂狗操作。
看门狗定时器内部有一个递减计数器,当该计数器递减为0的时候,就会自动重启控制器,如果我们写有这样的程序,该程序在定时器计数器递减为0之前,将其递减计数器重新设置一下(喂狗),那么就不会产生重启操作。假如机器设备出现异常情况下如死机,CPU执行出错,程序跑飞等情况,CPU就会陷入非正常的执行流程,就不会去执行重置计数器的程序,当计数器递减为0时,会产生复位控制器信号,机器就会重新启动,恢复正常执行流程。这样的设计原理就解决了很多环境恶劣的情况下,对服务器进行重启的任务。上面的重置倒计数的操作通常叫做“喂狗”。
为了避免看门狗带来的影响,简化系统,我们选择关闭看门狗。

上述代码向地址0x53000000(r1),也就是看门狗定时器控制寄存器(WTCON)写入了0x0000(r2),即16个0。
结合图\ref{看门狗},可以知道,这样配置的结果就是禁止了看门狗,系统也就不需要定时去喂狗了。

当然了,在比较正式的系统中,看门狗是必须要开启的,防止系统一直死机。这里由于我们只是在开发aCoral,所以暂时关闭。


\begin{figure}[H]
	\includegraphics[width=0.9\textwidth]{看门狗.png}
	\caption{看门狗定时器控制(WTCON)寄存器(手册第18章)}
	\label{看门狗}
\end{figure}
\subsection{关中断}
\begin{lstlisting}
	@ disable all interrupts
	mov	r1, #INT_CTL_BASE
	mov	r2, #0xffffffff
	str	r2, [r1, #oINTMSK]
	ldr	r2, =0x7ff
	str	r2, [r1, #oINTSUBMSK]
\end{lstlisting}

系统刚上电启动,这个时候,aCoral的中断系统还没有初始化,此时发生中断我们无法处理,所以要关闭中断。

代码中,立即数 INT\_CTL\_BASE = 0x4A000000 ,立即数 oINTMSK = 0x08 ,两个立即数相加的地址即指向中断屏蔽寄存器。中断屏蔽寄存器由 32 位组成,其每一位都都涉及一个中断源。如果某个指定为被设置为 1,则 CPU 不会去服务来自
相应中断源(请注意即使在这种情况中,SRCPND 寄存器的相应位也设置为 1)的中断请求。如果屏蔽位为 0,则
可以服务中断请求,如图\ref{中断屏蔽寄存器}。

oINTSUBMSK指向的次级中断屏蔽寄存器类似。

\begin{figure}[H]
	\includegraphics[width=0.9\textwidth]{中断屏蔽寄存器.png}
	\caption{中断屏蔽寄存器(手册第14章)}
	\label{中断屏蔽寄存器}
\end{figure}


\subsection{时钟初始化}
\begin{lstlisting}
	@ initialise system clocks
	mov	r1, #CLK_CTL_BASE
	mvn	r2, #0xff000000
	str	r2, [r1, #oLOCKTIME]

	mov	r1, #oCLKDIVN
	mov	r2, #M_DIVN
	str	r2, [r1, #oCLKDIVN]

	mrc	p15, 0, r1, c1, c0, 0	@ read ctrl register
	orr	r1, r1, #0xc0000000	@ Asynchronous
	mcr	p15, 0, r1, c1, c0, 0	@ write ctrl register

	mov	r1, #CLK_CTL_BASE
	ldr 	r2, =vMPLLCON	        @ clock user set
	str	r2, [r1, #oMPLLCON]
\end{lstlisting}

mini2440上电后,系统工作在板载晶振12MHz的频率下。这个频率比较低,系统的性能还没有完全得到发挥,所以需要激发一下潜能,使用一种叫做锁相环PLL的东西来倍频,加倍之后的频率再给CPU和板子上的其他设备使用。

mini2440有两个锁相环,MPLL和UPLL。
MPLL的输出频率直接给CPU使用,称为FCLK,同时经过上面初始化系统时钟后得到另外两个频率HCLK和PCLK,分别给板载高速硬件和低速外设使用。

将 M\_DIVN = 0x5 写入 oCLKDIVN + oCLKDIVN = 0x4C000014 寄存器后,三者的比例为FCLK:HCLK:PCLK=8:2:1,如图\ref{时钟分频控制寄存器}
\begin{figure}[H]
	\includegraphics[width=0.9\textwidth]{时钟分频控制寄存器.png}
	\caption{时钟分频控制寄存器(手册第7章)}
	\label{时钟分频控制寄存器}
\end{figure}

将 vMPLLCON = 0x1FC0021 写入 oMPLLCON + CLK\_CTL\_BASE = 0x4C000004 寄存器后,得到FCLK≈400MHz,再根据之前的比例得到HCLK 和 PCLK分别为100MHz和50MHz。
计算过程见图\ref{PLL 控制寄存器},其中Fin就是晶振的频率=12MHz。
\begin{figure}[H]
	\includegraphics[width=0.9\textwidth]{PLL 控制寄存器.png}
	\caption{PLL 控制寄存器(手册第7章)}
	\label{PLL 控制寄存器}
\end{figure}

UPLL的输出频率没有配置,直接给USB使用。


\subsection{存储空间初始化}
\begin{lstlisting}
	bl	memsetup
......
	@*************************************
	@ initialise the static memory
	@ set memory control registers
	@*************************************
memsetup:
	mov	r1, #MEM_CTL_BASE
	adrl	r2, mem_cfg_val
	add	r3, r1, #52
1:	ldr	r4, [r2], #4
	str	r4, [r1], #4
	cmp	r1, r3
	bne	1b
	mov	pc, lr
\end{lstlisting}

memsetup就是初始化一下mini2440的存储器(BANK0~BANK7),具体自行阅读《S3C2440中文手册》第5章。

\subsection{栈初始化}
\begin{lstlisting}
	bl      InitStacks
......
	@************************************
	@             堆栈初始化
	@************************************

InitStacks:
	mov r2,lr
	mrs	r0,cpsr
	bic	r0,r0,#MODE_MASK
	orr	r1,r0,#UND_MODE|NOINT
	msr	cpsr_cxsf,r1		@UndefMode
	ldr	sp,=UDF_stack		@ UndefStack=0x33FF_5C00

	orr	r1,r0,#ABT_MODE|NOINT
	msr	cpsr_cxsf,r1		@AbortMode
	ldr	sp,=ABT_stack		@ AbortStack=0x33FF_6000

	orr	r1,r0,#IRQ_MODE|NOINT
	msr	cpsr_cxsf,r1		@IRQMode
	ldr	sp,=IRQ_stack		@ IRQStack=0x33FF_7000

	orr	r1,r0,#FIQ_MODE|NOINT
	msr	cpsr_cxsf,r1		@FIQMode
	ldr	sp,=FIQ_stack		@ FIQStack=0x33FF_8000

	bic	r0,r0,#MODE_MASK|NOINT
	orr	r1,r0,#SVC_MODE
	msr	cpsr_cxsf,r1		@SVCMode
	ldr	sp,=SVC_stack		@ SVCStack=0x33FF_5800

	mrs     r0,cpsr
	bic     r0,r0,#MODE_MASK
	orr     r1,r0,#SYS_MODE|NOINT
	msr     cpsr_cxsf,r1    	@ userMode
	ldr     sp,=SYS_stack

	mov	pc,r2
\end{lstlisting}

InitStacks就是初始化一下arm处理器各个模式的栈。因为影子寄存器的存在,每个模式的sp寄存器都是独立的(除了系统和用户模式)。
具体的做法就是通过修改cpsr寄存器的[5:0]位,进入每一个模式并修改sp寄存器位该模式的栈底地址。
每个模式的栈底地址定义于链接脚本。链接脚本将在下一节进行讲解。cpsr寄存器各位编排如图\ref{程序状态寄存器格式}。
\begin{figure}[H]
	\includegraphics[width=0.9\textwidth]{程序状态寄存器格式.png}
	\caption{程序状态寄存器格式(手册第2章)}
	\label{程序状态寄存器格式}
\end{figure}

PS:关于影子寄存器(也称分组寄存器),请查看《S3C2440中文手册》第2章图2-3。


\subsection{自我拷贝(loader)}
\begin{lstlisting}
	adr  r0,__ENTRY
	ldr  r1,_text_start
	cmp  r0,r1
	blne copy_self  
...
copy_self:

	ldr	r1, =( (4<<28)|(2<<4)|(3<<2) )	/* address of Internal SRAM  0x4000002C*/
	mov	r0, #0		
	str	r0, [r1]


	mov	r1, #0x2c	/* address of men  0x0000002C*/
	ldr	r0, [r1]
	cmp	r0, #0
	bne	copy_from_rom
        
    ldr	r0, =(2440)
	ldr	r1, =( (4<<28)|(2<<4)|(3<<2) )
	str	r0, [r1]
	b       copy_from_nand 
\end{lstlisting}

这一步非常重要。如果aCoral是烧写在nor flash中,启动时是从0地址开始运行的,那aCoral怎么到内存SDRAM(起始地址为0x30000000)中运行呢?
这项任务是由aCoral自己来完成的,也就是我搬起了我自己。

我们先想一下,aCoral在上电运行的时候,怎么知道自己是在flash还是sdram中运行的呢?答案是看一下现在自己运行的地址是多少就行了,换句话说就是查看pc寄存器的值。
基于这种思路,aCoral使用了一种更严谨的做法:使用adr相对地址指令。

通过查看反汇编,
\begin{lstlisting}
	adr  r0,__ENTRY	
\end{lstlisting}

被汇编成

\begin{lstlisting}
	sub	r0, pc, #188
\end{lstlisting}

这就表示,无论程序在哪里运行,r0永远是程序的实际起始地址。比如说,当程序从0地址的flash开始运行,那r0的值就等于0;而当程序在0x30000000的sdram中运行时,
r0寄存器就等于0x30000000。

\begin{lstlisting}
	ldr  r1,_text_start
\end{lstlisting}

而\_text\_start处存放的是定值0x30000000,所以r1=0x30000000。这样通过比较r0与r1寄存器,如果r0=r1=0x30000000,就说明程序当前正在sdram中运行,就不要copy\_self;反之则需要。
copy\_self的过程自行阅读。

\subsection{bss段清零}
\begin{lstlisting}
	ldr  r0,_bss_start
	ldr  r1,_bss_end
	bl    mem_clear
.......
	@***********************************
	@ clear memory
	@ r0: start address
	@ r1: length
	@***********************************

mem_clear:
	mov r2,#0
1:	str r2,[r0],#4
	cmp r0,r1
	blt 1b
	mov pc,lr
\end{lstlisting}

bss段中的数据都是未初始化或者初始值为0的,而sdram本身是有一些随即初始值的,所以需要对bss段对应的内存进行清零。
mem\_clear的两个参数分别为bss段的起始地址和结束地址,都定义在链接脚本中。

\subsection{跳转至下一阶段}
\begin{lstlisting}
	ldr    pc,=acoral_start	
\end{lstlisting}

代码最终将寄存器pc设置为acoral\_start的值,表示CPU将跳转到acoral\_start函数处执行。acoral\_start函数位于
\begin{lstlisting}
 kernel\src\core.c
\end{lstlisting}

\section{启动-第二阶段}

硬件初始化完成后,就需要对系统的软件部分进行初始化。
现在我们来具体看一下,aCoral内核启动的第二阶段到底做了什么。

\subsection{acoral-start}
core.c中的acoral\_start()函数如下
\begin{lstlisting}
acoral_thread_t orig_thread;

void acoral_start(){
	orig_thread.console_id=ACORAL_DEV_ERR_ID;
	acoral_set_orig_thread(&orig_thread);
	/*板子初始化*/
	HAL_BOARD_INIT();

	/*内核模块初始化*/
	acoral_module_init();

	/*串口终端应该初始化好了,将根线程的终端id设置为串口终端*/
#ifdef CFG_DRIVER
	orig_thread.console_id=acoral_dev_open("console");
#endif

	/*主cpu开始函数*/
	acoral_core_cpu_start();
}
\end{lstlisting}

可以看到acoral\_start代码并不多,概括一下就是三件事:

(1)内核模块初始化。

这部分代码中,最重要的就是acoral\_module\_init()这个函数。
这个函数将初始化aCoral的各个模块,包括中断、内存、线程等嵌入式操作系统必需的模块。
\begin{lstlisting}
	void acoral_module_init(){
	/*中断系统初始化*/
	acoral_intr_sys_init();
	/*内存管理系统初始化*/
	acoral_mem_sys_init();
	/*资源管理系统初始化*/
	acoral_res_sys_init();
	/*驱动管理系统初始化*/
	/*线程管理系统初始化*/
	acoral_thread_sys_init();
	/*时钟管理系统初始化*/
	acoral_time_sys_init();
	/*事件管理系统初始化,这个必须要,因为内存管理系统用到了*/
	acoral_evt_sys_init();
	/*消息管理系统初始化*/
#ifdef CFG_DRIVER
	acoral_drv_sys_init();
#endif
}
\end{lstlisting}

(2)设置orig线程。

关于orig线程,有两个问题:这个线程是怎么创建的呢?为什么要设置一个orig线程呢?
第一个问题比较好回答,这个时候aCoral的线程模块还没有初始化,创建线程的那些函数都是用不来的,
所以就直接定义了一个acoral\_thread\_t 类型的orig\_thread。这个变量类型就是aCoral中的Task Control Block(TCB),在《aCoral内核手册》将会介绍。 
给成员console\_id赋值,并将这个orig线程设置为当前正在运行的线程,如代码:
\begin{lstlisting}
	orig_thread.console_id=ACORAL_DEV_ERR_ID;
	acoral_set_orig_thread(&orig_thread);
\end{lstlisting}

那么为什么要设置这个orig线程呢?我们先看一下console\_id被置为了什么。
\begin{lstlisting}
	orig_thread.console_id=acoral_dev_open("console");
\end{lstlisting}

这个orig线程的console\_id指向一个叫"console"的设备。
console控制台是干什么用的?console本质上就是UART串口,我们在串口工具的那个黑黑的界面上看到的信息,都是从串口线传过来的。
aCoral中我们经常见到的的函数acoral\_prints(),也是要把信息通过UART串口,显示到电脑的串口工具中。
打开这个console设备,本质上就是在注册的驱动中找到UART串口驱动。
话又说回来,那orig线程需要这个console干嘛呢?orig线程虽然不打印或接收信息,但是它需要这个console,
是为了之后由它创建的线程(子线程、孙子线程等等)都能继承到这个console,换句话说,它的子孙们都能找到从哪去接收信息,把信息打印打哪里去。
这点在acoral\_thread\_init()函数,也就是创建线程的过程中有体现。具体会在《aCoral内核手册》中介绍。

从这里也可以看出,orig是真正意义上的,aCoral中的第一个线程(祖宗)。


(3)调用acoral\_core\_cpu\_start()进行下一步初始化工作。

\subsection{acoral-core-cpu-start}
\begin{lstlisting}
	void acoral_core_cpu_start(){
	acoral_comm_policy_data_t data;
	/*创建空闲线程*/
	acoral_start_sched=false;
	data.cpu=acoral_current_cpu;
	data.prio=ACORAL_IDLE_PRIO;
	data.prio_type=ACORAL_ABSOLUTE_PRIO;
	idle_id=acoral_create_thread_ext(idle,IDLE_STACK_SIZE,NULL,"idle",NULL,ACORAL_SCHED_POLICY_COMM,&data);
	if(idle_id==-1)
		while(1);
	/*创建初始化线程,这个调用层次比较多,需要多谢堆栈*/
	data.prio=ACORAL_INIT_PRIO;
	/*动态堆栈*/
	init_id=acoral_create_thread_ext(init,ACORAL_TEST_STACK_SIZE,"in init","init",NULL,ACORAL_SCHED_POLICY_COMM,&data);
	if(init_id==-1)
		while(1);
	acoral_start_os();
}
\end{lstlisting}

这里,acoral\_core\_cpu\_start()概况起来也是做了三件事:

(1)创建idle线程。

orig线程的亲儿子,继承了orig线程的console。
idle线程又名守护线程,说白了就是,CPU没事干了,又不能真的闲下来,那就去执行idle线程,所以idle线程的优先级是最低的。
idle线程的工作函数如下,可以看到,idle线程就是一个无限循环,CPU没事干了就跑while循环摆烂。
\begin{lstlisting}
void idle(void *args){
	while(1){

	}
}
\end{lstlisting}

(2)创建init线程。

orig线程的亲儿子,继承了orig线程的console。
init线程做了很多初始化工作,包括时钟中断的注册、创建daem回收线程、shell线程初始化、插件初始化等。
这里解释一下shell。串口之所以能接收我们的指令并返回结果,就是shell这个线程在工作。
init线程的工作函数如下。
\begin{lstlisting}
void init(void *args){
	acoral_prints("in init spg");
	acoral_comm_policy_data_t data;
	acoral_ticks_init();
	/*ticks中断初始化函数*/
	acoral_start_sched=true;
	/*软件延时初始化函数*/
#ifdef CFG_SOFT_DELAY
	soft_delay_init();
#endif

	/*创建daem回收线程*/
  	acoral_init_list(&acoral_res_release_queue.head);
  	acoral_spin_init(&acoral_res_release_queue.head.lock);
	data.cpu=acoral_current_cpu;
	data.prio=ACORAL_DAEMON_PRIO;
	data.prio_type=ACORAL_ABSOLUTE_PRIO;
	daemon_id=acoral_create_thread_ext(daem,DAEM_STACK_SIZE,NULL,"daemon",NULL,ACORAL_SCHED_POLICY_COMM,&data);
	thread=(acoral_thread_t *)acoral_get_res_by_id(daemon_id);
	if(daemon_id==-1)
		while(1);
	/*应用级相关服务初始化,应用级不要使用延时函数,没有效果的*/
#ifdef CFG_SHELL
	acoral_shell_init();
#endif
	plugin_init();
	app_enter_policy_init();
	user_main();
#ifdef CFG_TEST
	test_init();
#endif
	app_exit_policy_init();
}
\end{lstlisting}


(3)调用acoral\_start\_os()正式开始运行系统。

最后一件事,无非就是初始化完成了,开始调度了,这个时候aCoral就正式开始运行了,一去不复返……





\input{chapter/c3_linkScript.tex}


\end{document}
