\chapter{概述}

\section{珊瑚(aCoral)简介}

珊瑚(aCoral)是电子科技大学信息与软件工程学院嵌入式智能计算研究团队开发的一款嵌入式实时操作系统,具有开源、高可配、高扩展性的特点。

珊瑚(aCoral)目前拥有单核(aCoral-Ⅰ)和多核(aCoral-Ⅱ)两个版本。本仓库中的文档将介绍珊瑚操作系统的单核版本aCoral-Ⅰ,使用的硬件平台为mini2440。
出于方便的目的,后续将单核版本的珊瑚简称为aCroal。单核版本的珊瑚(aCoral-Ⅰ)对于主流的开发平台都有支持,像 s3c2440,s3c2410,s3c44b0,lpc2313,lpc2200,stm3210。

aCoral 支持多线程模式,其最小配置生成的代码为 7K 左右,而配置文件系统、轻型 TCP/IP、GUI 后生成的代码仅有 300K 左右。

嵌入式操作操作系统一般都是实时的,但是如何做到强实时是一个很棘手的问题,为强实时计算密集型应用(航空电子、舰载电子„„)提供可靠运行支持
是 aCoral 开发的强力主线。目前 aCoral 提供了强实时内核机制(优先级位图法、优先级天花板协议、差分时间链、最多关中断时间)。与此同时,aCoral 还提供了
强实时调度策略:RM 调度算法,强实时确保策略也正在研究中。

aCoral 会像珊瑚一样成长......

\section{aCoral项目成员}
