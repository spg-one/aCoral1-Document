\chapter{绪\hspace{6pt}论}

\section{珊瑚(aCoral)简介}

珊瑚(aCoral)是电子科技大学信息与软件工程学院嵌入式智能计算研究团队开发的一款嵌入式实时操作系统,具有开源、高可配、高扩展性的特点。

珊瑚(aCoral)目前拥有单核(aCoral-Ⅰ)和多核(aCoral-Ⅱ)两个版本。本文档将介绍珊瑚操作系统的单核版本aCoral-Ⅰ。
出于方便的目的,下文中将全部简称为aCroal。单核版本的珊瑚(aCoral-Ⅰ)对于主流的开发平台都有支持,像 s3c2440,s3c2410,s3c44b0,lpc2313,lpc2200,stm3210。
本文档在硬件平台mini2440下编写。


\section{aCoral内核结构}
时域积分方程方法的研究始于上世纪60 年代,C.L.Bennet 等学者针对导体目
标的瞬态电磁散射问题提出了求解时域积分方程的时间步进(marching-on in-time,
MOT)算法。

\section{本文的主要贡献与创新}
本论文以时域积分方程时间步进算法的数值实现技术、后时稳定性问题以及两层平面波加速算法为重点研究内容,主要创新点与贡献如下:

\section{本论文的结构安排}
本文的章节结构安排如下:
